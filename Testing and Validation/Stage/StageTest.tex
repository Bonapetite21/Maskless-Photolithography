\documentclass[a4paper,12pt]{article}

% Packages
\usepackage[utf8]{inputenc}  % UTF-8 encoding
\usepackage{amsmath}         % Mathematical symbols
\usepackage{graphicx}        % Images
\usepackage{booktabs}        % Tables
\usepackage{hyperref}        % Hyperlinks
\usepackage{xcolor}          % Colored text
\usepackage{listings}        % Code listings

% Settings for Listings (code blocks)
\lstset{
    basicstyle=\ttfamily\footnotesize,
    frame=single,
    numbers=left,
    numberstyle=\tiny,
    breaklines=true,
    keywordstyle=\color{blue},
    commentstyle=\color{green!50!black},
    stringstyle=\color{red},
    tabsize=2
}

% Title and Author
\title{Testing and Validation Report}
\author{Your Name}
\date{\today}

\begin{document}

% Title Page
\maketitle
\tableofcontents
\newpage

% Section: Introduction
\section{Introduction}
This document outlines the testing and validation process for the project \textbf{[Project Name]}. It includes test objectives, methods, results, and conclusions.

\subsection{Purpose}
The purpose of this testing is to verify that the system meets the specified requirements and functions as intended.

% Section: Test Plan
\section{Test Plan}
\subsection{Test Environment}
The testing was conducted in the following environment:
\begin{itemize}
    \item Operating System: Ubuntu 22.04
    \item Hardware: Intel Core i7, 16GB RAM
    \item Tools: Python 3.10, Jupyter Notebook
    \item LaTeX: TeX Live 2023
\end{itemize}

\subsection{Test Cases}
The table below lists the test cases:

\begin{table}[h!]
    \centering
    \begin{tabular}{|l|p{8cm}|l|l|}
        \hline
        \textbf{Test ID} & \textbf{Description} & \textbf{Expected Result} & \textbf{Status} \\ \hline
        TC001 & Validate input file format & File is successfully parsed & Pass \\ \hline
        TC002 & Verify calculation accuracy & Results match theoretical values & Fail \\ \hline
        TC003 & Check output file generation & File is saved in the correct location & Pass \\ \hline
    \end{tabular}
    \caption{Test Cases}
    \label{tab:testcases}
\end{table}

% Section: Test Methods
\section{Test Methods}
\subsection{Input Validation}
Inputs were tested with valid, invalid, and edge cases to ensure proper handling. For example:

\begin{lstlisting}[language=Python, caption=Input Validation Code]
def validate_input(file_path):
    if not file_path.endswith('.csv'):
        raise ValueError("Invalid file format. Expected .csv")
\end{lstlisting}

\subsection{Output Verification}
The results were compared against known baselines to validate accuracy.

% Section: Results
\section{Results}
\subsection{Test Results Summary}
The following results were obtained:
\begin{itemize}
    \item Total Tests: 10
    \item Passed: 8
    \item Failed: 2
\end{itemize}

\subsection{Failed Test Analysis}
For test case TC002, the calculation mismatch was due to incorrect rounding. The expected result was 3.14, but the output was 3.14159.

\subsection{Figures}
Figure~\ref{fig:example} shows the comparison graph between expected and actual values.

\begin{figure}[h!]
    \centering
    \includegraphics[width=0.7\textwidth]{example-image} % Replace with your image file
    \caption{Comparison of Expected vs Actual Values}
    \label{fig:example}
\end{figure}

% Section: Conclusion
\section{Conclusion}
The testing and validation process identified minor issues that need resolution, but the overall system performs as expected.

% Appendix
\appendix
\section{Appendix}
\subsection{Code Snippets}
Below is the Python code used for testing:

\begin{lstlisting}[language=Python, caption=Sample Python Code]
import numpy as np

def calculate_area(radius):
    return np.pi * radius**2
\end{lstlisting}

\end{document}
